\documentclass[18pt,oneside,a4paper, titlepage]{article}

\usepackage[hidelinks]{hyperref}
\usepackage[pdftex]{graphicx}

\begin{document}
\begin{figure}[t]
	\centering
	\includegraphics[scale=0.35]{logo-polimi.png}
\end{figure}
\title{\textbf{myTaxiService}\\\textbf{P}roject \textbf{P}lan \textbf{D}ocument\\ A.Y. 2015/2016\\
	Politecnico di Milano\\ Version 1.0}	
\author{Cattaneo Michela Gaia, matr. 791685\\Barlocco Mattia, matr. 792735 }
\date{February 2, 2016}
\maketitle

\newpage
\tableofcontents

\newpage
\section{Introduction}
	\subsection{Purpose}
		This document represents the Project Plan Document.\\ It specifies the effort and cost estimation of the myTaxiService project, by using the Function Points and COCOMO algorithmic techniques. Moreover, there is the explanation of the organization of the resources and of the tasks needed.\\
		The intended audience of this document is the project stakeholders and the development team.
	\subsection{Definitions and abbreviations}
		\begin{itemize}
			\item \textbf{Definitions}
			\item[-] \textbf{Passenger}: a person who is registered in the application.
			\item[-] \textbf{Taxi driver}: a taxi driver who access the application with a specific ID.
			\item[-] \textbf{Request}: the request of a taxi in a certain area and position in the city made by a user.
			\item[-] \textbf{Reservation}: the reservation of a taxi in a certain area, place and time that can be made only by passengers.
	
			\item \textbf{Acronyms and abbreviations}
			\item[-] \textbf{RASD}: Requirement Analysis and Specification Document
			\item[-] \textbf{DD}: Design Document
			\item[-] \textbf{JVM}: Java Virtual Machine
			\item[-] \textbf{FP}: Function Points
			\item[-] \textbf{COCOMO}: COnstructive COst MOdel
			
		\end{itemize}
	\subsection{Reference documents}
		\begin{itemize}
			\item Project Description and Rules (\url{https://github.com/MichelaCattaneo/myTaxiService/blob/master/Project\%20Description\%20And\%20Rules.pdf})
			\item Requirements Analysis and Specification Document (\url{https://github.com/MichelaCattaneo/myTaxiService/blob/master/Deliveries/RASD_1.1.pdf})
			\item Design Document (\url{https://github.com/MichelaCattaneo/myTaxiService/blob/master/Deliveries/DD.pdf})
			\item Code Inspection Document(\url{https://github.com/MichelaCattaneo/myTaxiService/blob/master/Deliveries/CodeInspection.pdf})
			\item Integration Test Document (\url{https://github.com/MichelaCattaneo/myTaxiService/blob/master/Deliveries/ITPD.pdf})
		\end{itemize}

\newpage
\section{Effort and Cost Estimation}
	%Apply Function Points to the project size and then COCOMO to estimate effort and cost

	\subsection{Function Points}
	\subsection{COCOMO}
	
\newpage
\section{Task Scheduling}
	The tasks for our project are:
	\begin{itemize}
		\item \textbf{T1} RASD: this is the first task to do. It describes what exactly we have to do: requirements, interfaces, model of the Database and how our application should work.
		\item \textbf{T2} Design Document: it describes the architectural design of our system and it could be done at the same time of the T3.
		\item \textbf{T3} Database: based on the model descried on the RASD, in this task we create all the tables in the database.
		\item \textbf{T4} Interfaces: based on what we wrote on the RASD, in this task we create the interfaces for the client side.
		\item \textbf{T5} Requirements: in this task we develop the logic of the requirements on the server side (for example: the AccessManager component that manages the login and the registration, the RequestManager that manages the requests, the reservations and the notifications)
		\item \textbf{T6} Code Inspection: check if the code is well formatted.
		\item \textbf{T7} Code Inspection Document: in this task we write document of the code inspection.
		\item \textbf{T8} Integration Test Document: in this task we write the document related on the strategy we adopt for the integration test.
		\item \textbf{T9} Integration Test: based on what we wrote on the integration test document, in this task we test all the code of the project.
	\end{itemize}
	We suppose to work 8 hours per day to this project during the days.
	\begin{center}
		\centering
		\begin{tabular}{|c |c |c | c|}
			\hline \textbf{Task} & \textbf{Effort(person)} &\textbf{Duration (days)} & \textbf{Dependencies} \\
			\hline		T1 & 2 & 5 & -\\
			\hline		T2 & 1 & 5 & T1\\
			\hline		T3 & 1 & 3 & T1\\
			\hline		T4 & 1 & 10 & T2\\
			\hline		T5 & 1 & 10 & T3\\
			\hline		T6 & 2 & 10 & T4,T5\\
			\hline		T7 & 1 & 2 & T6\\
			\hline		T8 & 2 & 2 & T7\\
			\hline		T9 & 2 & 10 & T8\\
			\hline
		\end{tabular}
	\end{center}
	% Identify the tasks for your project and their schedule. Do so retrospectively, assuming that the project has started in October 2015, as it really happened.
	
\newpage
\section{Resources Allocation}
	% Allocate the resources (all memebers of your group) to the various tasks. In defining the allocation, take into account your actual availability for the project.

\newpage
\section{Risks and Recoveries}
	% Define the risks for the project, their relevance and the associated recovery actions.

\newpage
\section{Appendix}

	\subsection{Software and tool used}
		\begin{itemize}
			\item TeXstudio (\url{http://www.texstudio.org/}): to redact and to format this document.
		\end{itemize}
	
	\subsection{Hours of work}
		Time spent redacting this document:
		\begin{itemize}
			\item Cattaneo Michela Gaia: {\raise.17ex\hbox{$\scriptstyle\sim$}}\_ hours of work.
			\item Barlocco Mattia: {\raise.17ex\hbox{$\scriptstyle\sim$}}\_ hours of work.
		\end{itemize}

\end{document}