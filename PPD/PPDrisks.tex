\newpage
\section{Risks and Recoveries}
	% Define the risks for the project, their relevance and the associated recovery actions.
	Risk management is an important issue for the development of a project, as long as the risks can affect the project schedule or the quality of the product, increasing the costs and efforts that were previously estimated. \\ The strategy adopted is the proactive strategy, that follows fixed steps in order to manage the issues in advance. The first step consists in identifying potential problems and threats the product can be subject to. Then it is necessary to determine the recovery actions and all the ways to reduce the risks.\\ These are the risks that could be found:
	\begin{itemize}
		\item \textbf{Project risk.} This is a critical risk: it is possible that, going on in the implementation of the system, the requirements unexpectedly grows with respect to the one identified at the beginning. So it could be difficult to follow the project schedule and therefore the product is finished later than the estimated time. This will imply an increase in the efforts and in the costs.\\ In order to avoid this kind of problem, which has a high relevance, it is necessary to monitor every phase of the project, comparing it with existing projects. In case this is not enough, it can always be possible to release an early version of the project with less functionalities and then, when possible, release the complete version.
		\item \textbf{Market risk}. It is possible that, even though the project is good and efficient, it is difficult to sell because it is not what people needs in that moment, for example if there is already a similar product on the market.\\ There is not a real strategy to adopt against this kind of risks, in fact it is quite unpredictable and it depends on the market trends. In order to at least try to reduce the possibility of this kind of situations a study on the market could help.
		\item \textbf{Technology to be built.} This is a relevant risk relied to the technological part of the application: the most critical issues for an application like myTaxiService, that should be reliable and responsive, are data loss and performance issues.\\ In order to avoid the deterioration of the service, a particular care on this aspect must be taken in account. The server should always be on and available, in order to compute all the requests in a short amount of time and the algorithms computing the taxi queues and the taxi positioning should be implemented in the most efficient ways. In particular, to prevent data loss, the functioning protocol for the client-server data transmission should be adopted.
		\item \textbf{Staff experience.} It is possible that the lack of technical experience of the project developers affect the quality of the software and the schedule of the project.\\ In order to prevent this from happening it is necessary that the group is guided by someone with more experience or that the developers are able to follows existing guidelines of previous projects. This is not a really relevant risk, as the project developers can always take into account previous projects and material.
	\end{itemize}