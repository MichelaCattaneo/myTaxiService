\newpage
\section{Risks and Recoveries}
	% Define the risks for the project, their relevance and the associated recovery actions.
	Risk management is an important issue for the development of a project. \\ The first step consists in identifying potential problems and threats the product is subject to. Then it is necessary to determine the recovery actions and all the ways to reduce them.\\ These are the risks that could be found:
	\begin{itemize}
		\item \textbf{Project risk.} It could be difficult to follow the project schedule and therefore the product is finished later than the estimated time. This will imply an increase in the efforts and in the costs.\\ In order to avoid this kind of problem, which has a high relevance, it is necessary to monitor every phase of the project, comparing it with existing projects.
		\item \textbf{Business risk}. In particular market risk, in fact it is possible that, even tho the project is good and efficient, it is difficult to sell because it is not what people needs, for example if there is already a similar product preexisting.\\ There is not a real strategy to adopt against this kind of risk, in fact it is quite unpredictable and depends on market trends, in order to try to reduce the possibility of this kind of situations a study on the market could help.
		\item \textbf{Technology to be built.} This is a critical risk: it is possible that, going on in the implementation of the system, the requirements grows with respect to the one identified in the beginning. \\ A possible strategy is to overestimate the project in the early phases and anticipate all the possible scenarios.
		\item \textbf{Staff size and experience.} It is possible that the lack of technical experience of the project developers affect the quality of the software and the schedule of the project.\\ In order to prevent this from happening it is necessary that at least one component of the group has experience on previous projects or that the group follows existing guidelines. This is not a really relevant risk, as the project developers can always take into account previous projects and material.
	\end{itemize}
	It is important to be ready to recover from any potential risk, as long as they can affect the project schedule or the quality of the product, increasing the costs and efforts that were previously estimated.\\ The strategy adopted is the proactive strategy, that follows fixed steps in order to manage the risk in advance, instead of worrying about the problems only afterwards. In fact, it is important to take a forward-looking view, thinking about all the future issues that may arise.\\ Any possible threat is identified and analyzed, then there is an estimation of the probability and the impact of the damage. This analysis creates a ranking of all the risks, that allows to create a contingency plan that handles the higher priority and impact risks first.