\newpage
\section{Effort and Cost Estimation}
	%Apply Function Points to the project size and then COCOMO to estimate effort and cost

	\subsection{Function Points}
		Function Points is part of the algorithmic techniques for cost and estimation modeling. It bases on the assumption that the dimension of the software can be characterized by abstraction.\\
		This algorithmic technique is based on the complexity of five different entities: 
		\begin{itemize}
			\item number of input types: concern elementary operations that elaborate the data coming from the external environment.
			\item number of output types: concern elementary operations that generate the data for the external environment.
			\item number of inquiry types: concern input that controls the execution of the program and does not change the internal data structure, according to the query criteria.
			\item number of internal logic files (ILF): concern the internal data generated by the system.
			\item number of external interface files (EIF): concern the external interfaces to other systems or applications.
		\end{itemize}
		\vspace{0.5cm}
		The weights of this entities are showed in the table below.
		\vspace{0.9cm}
		\begin{center}
			\begin{tabular}{c c c c}
				\hline 	\textbf{Function Types} & & \textbf{Weight}  & \\[0.1cm]
				& Simple & Medium & Complex\\[0.1cm]
				\hline	N. Inputs & 3 & 4 & 6 \\[0.05cm]
				\hline	N. Outputs & 4 & 5 & 7 \\[0.05cm]
				\hline	N. Inquiry & 3 & 4 & 6 \\[0.05cm]
				\hline	N. ILF & 7 & 10 & 15 \\[0.05cm]
				\hline	N. EIF & 5 & 7 & 10 \\[0.05cm]
				\hline
			\end{tabular}
		\end{center}
		\newpage
		\subsubsection{External Inputs}
			The system allows the customer or the taxi driver to:
			\begin{itemize}
				 \item Sign up
				 \item Log in and log out
				 \item Request a taxi
				 \item Reserve a taxi
				 \item Choose the payment method
				 \item Manage the profile
				 \item Change status
				 \item Answer to taxi requests
			\end{itemize}
			All of these operations are rather simple and involve at most two entities, so a simple weight can be adopted.
			\begin{center}
				\textbf{Inputs: 9 x 3 = 27 FPs}
			\end{center}
		\subsubsection{External Outputs}
			The system sends to the customer information about the taxi driver arrival after the request. It sends to the taxi driver his position in the queue and the requests notification.\\
			The first two entities are complex, as they should get some information both from the ILF and the EIF, so a complex weight will be adopted. While, for the request notification, it is necessary to get information from two tables of the ILF, so it can be considered a medium complexity.
			\begin{center}
				\textbf{Outputs: 2 x 7 + 1 x 5 = 19 FPs}
			\end{center}
		\subsubsection{External Inquiry}
			The system allows the users and the taxi drivers to see their profile, which can be consider a simple information request.\\The passenger in particular can also see his previous requests and the taxi driver can see his position in the queue of his area and the previous notifications. The visualization of the position can be considered simple, while the visualization of the requests and notification requires to load a more consistent amount of data, so they will be considered medium-weighted.
			\begin{center}
				\textbf{Inquiries: 2 x 3 + 2 x 4 = 14 FPs}
			\end{center}
		\newpage
		\subsubsection{ILF}
			The application stores information about:
			\begin{itemize}
				\item passenger
				\item taxi driver
				\item city area
				\item request
				\item reservation
			\end{itemize}
			All these entities, except for the city area, have a simple structure in the database, in fact their table is composed of a few fields. Therefore we can decide to use a simple weight for all of them. As regards the city area table, it can be considered of medium complexity because it has to store also the information about the taxi queues.
			\begin{center}
				\textbf{ILF: 4 x 7 + 1 x 10= 38 FPs}
			\end{center}
		\subsubsection{EIF}
			The application manages the interaction with three external systems:
			\begin{itemize}
				\item Google Maps, in order to determine the position of the taxi drivers and of the users.
				\item Facebook, in order to get the user personal information if he wants to log in with his Facebook account.
				\item Google+, in  order to get the user personal information if he wants to log in with his Google+ account.
			\end{itemize}
			The position of all the different users can be considered a medium complexity entity, so we will adopt a medium weight. As regards the other two entities, Facebook and Google+, they can be considered simple weighted, as they have a simple structure.\\
			\begin{center}
				\textbf{EIF: 1 x 7 + 2 x 5 = 17 FPs}
			\end{center}
		\subsubsection{Result}
			The total number of function points is:
			\vspace{0.5cm}
			\begin{center}
				\begin{tabular}{c c c c}
					\hline 	\textbf{Function Types} &	&	& \textbf{FPs} \\[0.1cm]
					\hline	N. Inputs & & & 27 \\[0.05cm]
						N. Outputs & & & 19  \\[0.05cm]
						N. Inquiry & & & 14  \\[0.05cm]
						N. ILF & & & 38  \\[0.05cm]
						N. EIF & & & 17  \\[0.05cm]
					\hline Total FPs & & & 115\\[0.05cm]
					\hline
				\end{tabular}
			\end{center}
			\vspace{0.5cm}
	\newpage
	\subsection{COCOMO}
		COCOMO is a cost estimation model. It is based on the FP previously calculated and on the lines of code of the project, that, along with the scale drivers, help to determine the general effort, duration and number of people needed for the project.\\
		In order to calculate the Source Lines Of Code (SLOC), it necessary to know how much SLOC per FP are needed in average for a J2EE project. This information can be found on the QSM website (\url{http://www.qsm.com/resources/function-point-languages-table}), where we can find the factor 46. Therefore the estimated number of lines of code is:
		\vspace{0.5cm}	
		\begin{center}
			\begin{math}
				SLOC = FP * 46 = 115 * 46 = 5290
			\end{math}
		\end{center}
		\vspace{0.5cm}
		We can consider the "Nominal" values of Cost Drivers (EAF = 1.00) and Scale Drivers (E = 1.0997) in order to calculate the effort, inserting this values in the formula:
		\vspace{0.5cm}
		\begin{center}
			\begin{math}
				effort = 2.94*EAF*(KSLOC)^{E}
			\end{math}
		\end{center}
		\begin{center}
			\begin{math}
				effort = 2.94*1.00*(5.29)^{1.0997} = 18.36 Person/months
			\end{math}
		\end{center}
		\vspace{0.5cm}
		The duration of the project can be found with this formula, considering the exponent E = 0.3179:
		\vspace{0.5cm}
		\begin{center}
			\begin{math}
				Duration = 3.67*(effort)^{E}
			\end{math}
		\end{center}
		\begin{center}
			\begin{math}
				Duration = 3.67*(18.36)^{0.3179} = 9.25 Months
			\end{math}
		\end{center}
		\vspace{0.5cm}
		Now it is possible to calculate	the number of people needed for this project.
		\vspace{0.5cm}
		\begin{center}
			\begin{math}
				N. people = effort / Duration
			\end{math}
		\end{center}
		\begin{center}
			\begin{math}
				N. people = 18.36 / 9.25 = 1.98 people
			\end{math}
		\end{center}
		\vspace{0.5cm}
	\subsection{Conclusions}
		In the Function Points part, we have overestimated some weight of the entities in order to reach an estimation that allows to finish in time or before the expected time. As regards the COCOMO results, the number of people coincide with the reality and the duration of the project is reasonable, considering the number of hours we have calculated for the documentation and the time needed for the code and testing.