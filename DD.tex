\documentclass[18pt,oneside,a4paper, titlepage]{article}

\usepackage[hidelinks]{hyperref}
\usepackage[pdftex]{graphicx}
\usepackage{amsmath}
\usepackage{algorithm}
\usepackage{algorithmicx}
\usepackage{algpseudocode}

\begin{document}
\begin{figure}[t]
	\centering
	\includegraphics[scale=0.35]{logo-polimi.png}
\end{figure}
\title{\textbf{myTaxiService}\\\textbf{D}esign \textbf{D}ocument\\ A.Y. 2015/2016\\
	Politecnico di Milano \\ Version 1.0}	
\author{Cattaneo Michela Gaia, matr. 791685\\Barlocco Mattia, matr. 792735 }
\date{December 4, 2015}
\maketitle

\newpage
	\tableofcontents
% ADD DOMAIN PROPERTY: ALMENO UN TAXI ACCETTA LA RICHIESTA. ????????????????????????????
\newpage
	\documentclass[18pt,oneside,a4paper, titlepage]{article}	
	
\begin{document}


\newpage
\section{Introduction}		
	\subsection{Purpose}
		This document represents the Design Document, which main goal is to describe the overall system architecture and to show the technical design decisions made, with the support of schema and diagrams.\\ It delineates how the software system will be structured in order to satisfy the requirements defined in the Requirement Analysis and Specification Document (RASD), each one translated into a representation of components, interfaces, and data necessary in the implementation phase.
		\\This document is addressed to the project development teams, technical architects, database designers and testers, as long as it is has useful guidelines and a specific description of the design implementation of the system.
		
	\subsection{Scope}
		This document outlines the implementation and the architectural design details of the myTaxiService system.
	\subsection{Definitions, Acronyms, Abbreviations}
		\begin{itemize}
			\item \textbf{Definitions}
				\item[-] \textbf{User}: a person who requests a service from the system. It can be a visitor or a passenger.
				\item[-] \textbf{Visitor}: a person who is not registered in the application.
				\item[-] \textbf{Passenger}: a person who is registered in the application.
				\item[-] \textbf{Taxi driver}: a taxi driver who access the application with a specific ID.
				\item[-] \textbf{Request}: the request of a taxi in a certain area and position in the city made by a user.
				\item[-] \textbf{Reservation}: the reservation of a taxi in a certain area, place and time that can be made only by passengers.
			\item \textbf{Acronyms and abbreviations}
				\item[-] \textbf{RASD}: Requirement Analysis and Specification Document
				\item[-] \textbf{DBMS}: Database Management System
				\item[-] \textbf{JEE}: Java Enterprise Edition
				\item[-] \textbf{API}: Application Programming Interface
				\item[-] \textbf{E-R}: Entity-Relationship
				\item[-] \textbf{BCE}: Boundary-Control-Entity
				\item[-] \textbf{UML}: Unified Modeling Language
				\item[-] \textbf{UX}: User eXperience
				\item[-] \textbf{HTML}: HyperText Markup Language
				\item[-] \textbf{HTTP}: HyperText Transfer Protocol
				\item[-] \textbf{MVC}: Model View Controller
			
		\end{itemize}
	\subsection{Reference documents}
		\begin{itemize}
			\item myTaxiService Requirement Analysis and Specification Document (RASD)
		\end{itemize}
	\subsection{Document structure}
		This document is divided in six main parts:
		\begin{itemize}
			\item Introduction: this section describes the document in general and its purpose.
			\item Architectural Design: this section specifies the architectural design part, giving information about the components involved.
			\item Algorithm Design: this section provides a general description of the main algorithms used during implementation.
			\item User Interface Design: this section outlines an overview on how the user interfaces of the system will look like. 
			\item Requirements Traceability: this section explains the connection between the requirements already defined in the RASD and the design elements introduced in this document.
			\item Appendix: this section provides information about the tools used to redact this document and how many hours of work each author has spent. 
		\end{itemize}


\end{document}
		
	\section{Architectural Design}
		\subsection{Overview}
		\subsection{High level components and their interaction}
		% here you can introduce the high level components of your architecture (in our basic example in the slides about design you find these in slide 7) and describe the main interaction between them (no details here. You can say why some components talk to each other, why, if the communication is synchronous or asynchronous, any other info you think is useful at this point). 
		\subsection{Component view}
		% here you have a refinement of what you have in Section 4.B and identify sub-components. For instance, the diagram in slide 6 could be a diagram showing a  component view
		
		\subsection{Deployment view}
		% this is what you have in slide 8, that is, the identification of the artifact that need to be deployed to have the system working
		\subsection{Runtime view}
		%	You can use sequence diagrams to describe the way components interact to accomplish specific tasks typically related to your use cases
		% this is what you have in slide 9 plus sequence diagrams describing the way components behave in order to accomplish a certain activity
		\subsection{Component interfaces}
		% here you define the interfaces of your components, that is, which operations they offer to the external world, their meaning, any input and output parameter (name, possible set of values/type)
		\subsection{Selected architectural styles and patterns}
		%	Please explain which style/patterns you used, why and how
			
		\subsection{Other design decisions}
		% Section 4.G and 4.H are meant to include any explanation of the choices you have made and of their rationale. 
		
	\documentclass[18pt,oneside,a4paper, titlepage]{article}

\usepackage{amsmath}
\usepackage{algorithm}
\usepackage{algorithmicx}
\usepackage{algpseudocode}
\begin{document}
	\newpage
	
	\section{Algorithm Design}
	%	Focus on the definition of the most relevant algorithmic part of your project
		The most relevant algorithm of the myTaxiService application is the one implementing the queue management. It is important to optimize its implementation as long as it is the most complex and the one that distinguishes the system.\\
		Another significant algorithm is the one that manages the forwarding of the requests of the users to the taxi drivers.
		\begin{itemize}
			\item \textbf{Initialization.} The class which manages the areas distribution and the queues of the taxi driver also deals with the initialization of the queues of the taxis. It starts assigning to every area as many taxi as the average number of the requests expected in that area, based on the statistics. Once all the areas have been filled with their optimal number of taxis, the remaining taxi drivers are sent to the areas that are most needy.\\The "needyArea" function, in fact, returns the index of the most needy area at the moment, calculating the area that has the maximum difference between the maximum number of requests recorded and the average of requests.\\
			 \\
			
			
			\begin{algorithm}
				\caption{Initialization}
				\begin{algorithmic}[1]
					\Procedure{initQueues}{TaxiDriver[] taxiDrivers, Area[] areas}
					\State $i \gets \textit{0}$
					\State $j \gets \textit{0}$
					\For {$i < \textit{areas.size()}$}
						\State $k \gets \textit{0}$
						\For {$k < \textit{areas[i].getAverage()}$}
								\State $areas[i].\text{addToQueue}(taxiDrivers[j])$
								\State $j \gets j+1$.
								\EndFor
						\EndFor
					\If{$j < \textit{taxiDrivers.size()}$ }
						\For {$j < \textit{taxiDrivers.size()}$}
							\State $areas[\text{needyArea} (areas)].\text{addInQueue}(taxiDrivers[j])$
						\EndFor
					\EndIf
					\EndProcedure
				\end{algorithmic}
			\end{algorithm}
			
			COMPLEXITY: O(T) with T = number of taxis.
			\newpage
			\item \textbf{Manage requests.} After the initialization, the taxi drivers are able to take requests. If they accept the request, the function sets their availability to false and stops monitoring their position. It is also possible that a taxi driver declines the request or is not able to give an answer to the server. In this case, the first taxi driver is moved to the last position of the queue and forwards the request to the second of the queue, who is now moved to the first position. This action is iterated in the queue of that area until a taxi accept the request. After accepting the call, the taxi driver is set unavailable and he is deleted from the queue, as it is likely that he will end up in another area.\\
			 \\
			
				\begin{algorithm}
					\caption{Manage requests}
					\begin{algorithmic}[1]
						\Procedure{manageRequest}{Position pos, Area area}
						\State $answer \gets \text{"no"}$
						\State $timer \gets \text{new Timer(60)}$
						\While {$answer = \text{"no"} \text{  }||\text{  } answer = \text{"timeOut"}$}
							\State $\text{sendRequest}(area.\text{getFirstOfQueue()}, pos)$
							\State $answer \gets \text{waitAnswer}(timer)$
							\If{$answer = \text{"no"} \text{ }||\text{  } answer = \text{"timeOut"}$}
								\State $area.\text{moveToEndOfQueue()}$
							\EndIf
						\EndWhile
						\State $area.\text{getFirstOfQueue()}.setUnavailable()$
						\State $area.\text{deleteFirstFromQueue()}$
						\EndProcedure
					\end{algorithmic}
				\end{algorithm}
			COMPLEXITY: O(A+Q) with A = number of areas and Q = number of taxi enqueued in that area.
			\newpage
			\item \textbf{Manage queues.} The areas of the city are represented by a graph with an array of adjacencies and the queue of the taxi drivers, whose position is within its boundaries, is assigned to each area.\\ It is a useful representation in order to decide how to distribute the taxis, in fact it is possible that an area is occupied by a number of taxi that is equal to the threshold of the maximum taxis that can be present in that area. In this case the system does not put a recently arrived taxi in the queue of that area, but advise the taxi driver that he will be moved to an adjacent area that is needy. If there are not needy adjacent areas, he will be assigned to a random area and if it \\
			
			
			 As it is necessary to guarantee that there should always be at least one taxi driver in each area, when it occurs that the last taxi driver leaves an area, the system instantly notifies the last taxi driver of the queue that he has to move from the most populated adjacent area to the needy area.
			termina sicuramente dentro il while perchè i taxi non sono tutti massimi.
			
			
			\begin{algorithm}
				\caption{Manage queues}
				\begin{algorithmic}[1]
					\Procedure{manageQueue}{Area[] area, TaxiDriver taxiDriver}
					\State $List<Area> \text{ } queue$
					\State $Area \text{ } tmp$
					\For {$ i < area.size() $}
						\If {$ taxiDriver.position \text{ is in } area[i]$}
							\State $ startingArea \gets i $
						\EndIf
					\EndFor
					\If {$ area[startingArea].getNumberOfTaxi() < MAX $}
						\State $ area[startingArea].addToQueue(taxiDriver)$
					
					\Else
						\For {$ i < area.sie() $}
							\State $area[i].distance \gets INFINITY $
						\EndFor
						\State $area[startingArea].distance \gets 0$
						\State $ queue.enqueue(area[startingArea])$
						\While {$ \text{!}queue.isEmpty()$}
							\State $ tmp \gets queue.dequeue() $
							\For {$i < tmp.adjacencies.size()$}
								\If{$ tmp.adjacencies[i].getNumberOfTaxi() < avg[i]$}
									\State $tmp.adjacencies[i].addToQueue(taxiDriver)$
									\State \textbf{return} 
								\EndIf
							\EndFor
							\For{$ i < tmp.adjacencies.size()$}
								\If{$ tmp.adjacencies[i].getNumberOfTaxi() < MAX$}
									\State $tmp.asjacencies[i].addToQueue(taxiDriver)$
									\State \textbf{return} 
								\EndIf
								\If{$tmp.adjacencies[i].distance = INFINITY$}
									\State $tmp.adjacencies[i].distance \gets tmp.distance + 1$
									\State $ queue.enqueue(tmp.adjacencies[i])$
								\EndIf
							\EndFor
						\EndWhile
					
					
					\EndIf
					\EndProcedure
				\end{algorithmic}
			\end{algorithm}
			COMPLEXITY: O(A) with A = number of areas.
		\end{itemize}
		
		%the complexity of the algorithms
	ADD DOMAIN PROPERTY: ALMENO UN TAXI ACCETTA LA RICHIESTA.
	
	
\end{document}
\newpage	
	\section{User Interface Design}
		The user interface design has been already specified in the section 3.1.1 of the RASD, where all the user interfaces are described with the support of mockups.
	%	Provide an overview on how the user interfaces of your system will look like. If you have already included this part in the RASD you can simply refer to what you have already done, possibly providing here some extensions if applicable.
		
	\section{Requirements Traceability}
	%	Explain how the requirements you have defined in the RASD map into the design elements that you have defined in this document.
\newpage
	\section{Appendix}
		\subsection{Software and tools used}
				\begin{itemize}
					\item TeXstudio (\url{http://www.texstudio.org/}): to redact and to format this document.
					\item draw.io (\url{https://www.draw.io/}):to create all the diagrams.
				\end{itemize}
		\subsection{Hours of work}
			Time spent redacting this document:
			\begin{itemize}
				\item Cattaneo Michela Gaia: {\raise.17ex\hbox{$\scriptstyle\sim$}}1 hours of work.
				\item Barlocco Mattia: {\raise.17ex\hbox{$\scriptstyle\sim$}}1 hours of work.
			\end{itemize}
		
		
		
\end{document}