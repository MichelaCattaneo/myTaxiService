\documentclass[18pt,oneside,a4paper, titlepage]{article}	
	
\begin{document}


\newpage
\section{Introduction}		
	\subsection{Purpose}
		This document represents the Design Document, which main goal is to describe the overall system architecture and to show the technical design decisions made, with the support of schema and diagrams.\\ It delineates how the software system will be structured in order to satisfy the requirements defined in the Requirement Analysis and Specification Document (RASD), each one translated into a representation of components, interfaces, and data necessary in the implementation phase.
		\\This document is addressed to the project development teams, technical architects, database designers and testers, as long as it is has useful guidelines and a specific description of the design implementation of the system.
		
	\subsection{Scope}
		This document outlines the implementation and the architectural design details of the myTaxiService system.
	\subsection{Definitions, Acronyms, Abbreviations}
		\begin{itemize}
			\item \textbf{Definitions}
				\item[-] \textbf{User}: a person who requests a service from the system. It can be a visitor or a passenger.
				\item[-] \textbf{Visitor}: a person who is not registered in the application.
				\item[-] \textbf{Passenger}: a person who is registered in the application.
				\item[-] \textbf{Taxi driver}: a taxi driver who access the application with a specific ID.
				\item[-] \textbf{Request}: the request of a taxi in a certain area and position in the city made by a user.
				\item[-] \textbf{Reservation}: the reservation of a taxi in a certain area, place and time that can be made only by passengers.
			\item \textbf{Acronyms and abbreviations}
				\item[-] \textbf{RASD}: Requirement Analysis and Specification Document
				\item[-] \textbf{DBMS}: Database Management System
				\item[-] \textbf{JEE}: Java Enterprise Edition
				\item[-] \textbf{API}: Application Programming Interface
				\item[-] \textbf{E-R}: Entity-Relationship
				\item[-] \textbf{BCE}: Boundary-Control-Entity
				\item[-] \textbf{UML}: Unified Modeling Language
				\item[-] \textbf{UX}: User eXperience
				\item[-] \textbf{HTML}: HyperText Markup Language
				\item[-] \textbf{HTTP}: HyperText Transfer Protocol
				\item[-] \textbf{MVC}: Model View Controller
			
		\end{itemize}
	\subsection{Reference documents}
		\begin{itemize}
			\item myTaxiService Requirement Analysis and Specification Document (RASD)
		\end{itemize}
	\subsection{Document structure}
		This document is divided in six main parts:
		\begin{itemize}
			\item Introduction: this section describes the document in general and its purpose.
			\item Architectural Design: this section specifies the architectural design part, giving information about the components involved.
			\item Algorithm Design: this section provides a general description of the main algorithms used during implementation.
			\item User Interface Design: this section outlines an overview on how the user interfaces of the system will look like. 
			\item Requirements Traceability: this section explains the connection between the requirements already defined in the RASD and the design elements introduced in this document.
			\item Appendix: this section provides information about the tools used to redact this document and how many hours of work each author has spent. 
		\end{itemize}


\end{document}